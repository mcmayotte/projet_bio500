\documentclass[9pt,twocolumn,twoside,]{pnas-new}

% Use the lineno option to display guide line numbers if required.
% Note that the use of elements such as single-column equations
% may affect the guide line number alignment.


\usepackage[T1]{fontenc}
\usepackage[utf8]{inputenc}

% tightlist command for lists without linebreak
\providecommand{\tightlist}{%
  \setlength{\itemsep}{0pt}\setlength{\parskip}{0pt}}


% Pandoc citation processing
\newlength{\cslhangindent}
\setlength{\cslhangindent}{1.5em}
\newlength{\csllabelwidth}
\setlength{\csllabelwidth}{3em}
\newlength{\cslentryspacingunit} % times entry-spacing
\setlength{\cslentryspacingunit}{\parskip}
% for Pandoc 2.8 to 2.10.1
\newenvironment{cslreferences}%
  {}%
  {\par}
% For Pandoc 2.11+
\newenvironment{CSLReferences}[2] % #1 hanging-ident, #2 entry spacing
 {% don't indent paragraphs
  \setlength{\parindent}{0pt}
  % turn on hanging indent if param 1 is 1
  \ifodd #1
  \let\oldpar\par
  \def\par{\hangindent=\cslhangindent\oldpar}
  \fi
  % set entry spacing
  \setlength{\parskip}{#2\cslentryspacingunit}
 }%
 {}
\usepackage{calc}
\newcommand{\CSLBlock}[1]{#1\hfill\break}
\newcommand{\CSLLeftMargin}[1]{\parbox[t]{\csllabelwidth}{#1}}
\newcommand{\CSLRightInline}[1]{\parbox[t]{\linewidth - \csllabelwidth}{#1}\break}
\newcommand{\CSLIndent}[1]{\hspace{\cslhangindent}#1}


\templatetype{pnasresearcharticle}  % Choose template

\title{Analyse comparative des réseaux de collaboration entre les
étudiants du cours BIO500 et les réseaux écologiques}

\author[a,1]{Ariane Barrette}
\author[a]{Mia Carrière}
\author[a]{Laurie-Anne Cournoyer}
\author[a]{Marie-Claude Mayotte}

  \affil[a]{Département de biologie, Faculté des sciences, Université de
Sherbrooke, Sherbrooke, Québec, J1K 2R1}


% Please give the surname of the lead author for the running footer
\leadauthor{}

% Please add here a significance statement to explain the relevance of your work
\significancestatement{}


\authorcontributions{}



\correspondingauthor{\textsuperscript{1} To whom correspondence should
be addressed. E-mail:
\href{mailto:ariane.barrette@usherbrooke.ca}{\nolinkurl{ariane.barrette@usherbrooke.ca}}}

% Keywords are not mandatory, but authors are strongly encouraged to provide them. If provided, please include two to five keywords, separated by the pipe symbol, e.g:
 \keywords{  Réseau écologique |  Interactions entre les
collaborations  } 

\begin{abstract}
Les réseaux écologiques peuvent être utilisés dans le but de comprendre
les interactions entre les différentes espèces d'une communauté
écologique. Ainsi, cet article vise à comparer les réseaux de
collaboration entre les étudiants du cours de méthodes en écologie
computationnelle (BIO500) à ceux des réseaux écologiques. Trois
questions de recherches furent élaborées afin d'atteindre l'objectif
principale de cet article. Une campagne de collecte de données avec la
coopération de tous les étudiants du cours a d'abord été créé afin de
concevoir trois bases de données (collaboration, étudiant et cours). Une
analyse des trois bases de données a été effectué afin de réaliser trois
figures à l'aide de différentes librairies dans RStudio. Les résultats
montrent que la centralité est un concept important dans les réseaux
écologiques et peut également être utilisés dans le contexte scolaire.
En effet, elle permet d'identifier les étudiants « clés » dans le réseau
de collaboration des étudiants. De plus, les résultats permettent de
constater que les étudiants ont interagis davantage avec les zones
centrales comme Montréal, Sherbrooke ou Trois-Rivières quel que soit
leur région administrative d'origine.
\end{abstract}

\dates{This manuscript was compiled on \today}
\doi{\url{www.pnas.org/cgi/doi/10.1073/pnas.XXXXXXXXXX}}

\begin{document}

% Optional adjustment to line up main text (after abstract) of first page with line numbers, when using both lineno and twocolumn options.
% You should only change this length when you've finalised the article contents.
\verticaladjustment{-2pt}



\maketitle
\thispagestyle{firststyle}
\ifthenelse{\boolean{shortarticle}}{\ifthenelse{\boolean{singlecolumn}}{\abscontentformatted}{\abscontent}}{}

% If your first paragraph (i.e. with the \dropcap) contains a list environment (quote, quotation, theorem, definition, enumerate, itemize...), the line after the list may have some extra indentation. If this is the case, add \parshape=0 to the end of the list environment.

\acknow{Please include your acknowledgments here, set in a single
paragraph. Please do not include any acknowledgments in the Supporting
Information, or anywhere else in the manuscript.}

This PNAS journal template is provided to help you write your work in
the correct journal format. Instructions for use are provided below.

Note: please start your introduction without including the word
``Introduction'' as a section heading (except for math articles in the
Physical Sciences section); this heading is implied in the first
paragraphs.

\hypertarget{introduction}{%
\section{Introduction}\label{introduction}}

Dans une communauté écologique, différentes espèces interagissent
ensemble. L'interaction entre ces espèces peut mener à produire
différents modèles de réseau. En effet, au sein des réseaux écologiques,
l'organisation et la position des espèces dans la communauté peuvent
être déterminées (1). Les réseaux écologiques des espèces peuvent
également s'appliquer à d'autres contextes. L'objectif principal de cet
article vise à comparer les réseaux de collaboration entre les étudiants
du cours de méthodes en écologie computationnelle à ceux des réseaux
écologiques. L'atteinte de cet objectif permettra de mieux comprendre
les différences entre les propriétés des réseaux des étudiants en
écologie et celles des réseaux écologiques. Ainsi, pour notre projet,
trois questions de recherches ont été élaborées : (1) « Peut-on observer
un patron de centralité entre les étudiants d'une classe en analysant
leurs interactions dans les travaux scolaires ? », (2) « Retrouve-t-on
des patrons d'interactions entre les élèves basé sur leur formation
préalable et leur année d'admission au baccalauréat ? », (3) « Comment
les relations entre les élèves d'une classe varient-elles en fonction de
leur région administrative ? ». Afin de répondre à ces différentes
questions de recherches, une analyse collaborative sur la plateforme
GitHub a été réalisé à l'aide de Rstudio.

\hypertarget{muxe9thodes}{%
\section{Méthodes}\label{muxe9thodes}}

Afin de répondre à ces questions de recherche, une campagne de collecte
de données a été réalisée en collaboration avec tous les étudiants du
cours de méthodes computationnelles afin de créer trois bases de données
(collaboration, étudiant et cours). Afin d'analyser ces données, un
projet collaboratif sur la plateforme GitHub a été créé. À partir de
cette plateforme et par l'entremise de Rstudio, le nettoyage des
données, la création des figures et du target ont été réalisés.
Plusieurs librairies ont été utilisés pour créer les différentes figures
: leaflet, leaflegend, htmlwidgets, webshot, ggplot2, gplots, igraph et
fields. En ce qui concerne le target, c'est la librairie RSQLite qui a
été utilisée. Pour terminer, l'article a été réalisé à l'aide de
Rmarkdown. \# Résultats

\hypertarget{discussion}{%
\section{Discussion}\label{discussion}}

\hypertarget{centralituxe9}{%
\subsection{Centralité}\label{centralituxe9}}

La centralité est un concept important dans les réseaux écologiques
puisqu'elle est souvent utiliser afin d'identifier les espèces « clés »
du système (2). En effet, selon ce principe, les espèces participants
dans plus de chaines sont plus susceptibles d'affecter l'abondance des
autres espèces (2). De plus, elle peut apporter de l'information pour
prédire quelle espèce, si elle venait à disparaître, aurait le plus
grand impact sur la communauté (2). Toutefois, il faut apporter la
nuance que la centralité ne parvient pas à identifier les espèces qui
sont moins connectés, mais qui ont tout de même une grande capacité de
contrôle sur la communauté (2). Dans le cas de cette expérience, il est
premièrement possible d'observer un patron de centralité entre les
étudiants en analysant leurs interactions dans les travaux scolaires. En
effet, les étudiants ayant collaborés avec un grand nombre de personnes
différentes sont représentés comme étant des étudiants centraux. En
interagissant avec une grande partie du réseau, ces étudiants ont un
impact significatif sur l'abondance des autres élèves, qui dans ce
contexte peut être représentée par leurs notes scolaires, établissant
ainsi un lien avec les systèmes écologiques. Par ailleurs, ce réseau
prend seulement en compte les interactions entre les élèves de la
classe. Toutefois, plusieurs élèves ont interagi, au cours de leur
baccalauréat, avec des étudiants ne se sont pas dans ce cours. Par
conséquent, ces étudiants font donc partie d'un autre réseau et la
centralité ne nous permet donc pas d'évaluer adéquatement leur
importance.

\hypertarget{patrons-dinteractions}{%
\subsection{Patrons d'interactions}\label{patrons-dinteractions}}

Tous les réseaux écologiques, même les plus hétérogènes, présentent
certaines structures et regroupements d'interactions entre les nœuds (1,
3). En fait, il existe 13 patrons possibles pour une interaction à trois
nœuds chacun représentant une relation différente (1, 3). Par exemple,
il y a la concurrence entre A et B pour la ressource partagée C (A
-\textgreater{} C \textless- B) ou une chaine alimentaire où A prédate B
et B prédate C (A -\textgreater{} B -\textgreater{} C) (1, 3). En se
basant sur la formation préalable des étudiants, on peut observer que
les étudiants ayant une formation universitaire sont plus susceptible de
travailler avec un plus grand nombre de personnes différentes que les
personnes ayant une formation technicienne. De plus, les étudiants qui
sont rentrés au baccalauréat en écologie en hiver 2019 ont travaillé
avec un plus grand nombre de personnes que ceux des autres sessions
d'admission et cela malgré qu'ils soient beaucoup moins que les
étudiants à être rentrés à l'automne 2020. Cela est assez logique
puisque ces étudiants ont un parcours différent et par conséquent, ils
doivent souvent collaborés avec d'autres cohortes. Il est donc possible
d'observer certains patrons d'interactions entre les élèves.

Toutefois, ces résultats sont surtout préliminaires puisqu'ils ne
tiennent pas compte de la proportion d'étudiants dans chaque catégorie.
Par exemple, comme il y a moins d'étudiants qui ont une formation
technicienne que pré-universitaire, il est logique qu'il possède un
nombre d'interactions inférieurs aux autres catégories. Pour avoir un
meilleur aperçu des regroupements, il faudrait pondérer les moyens de
chaque catégorie.

\hypertarget{connectivituxe9-des-ruxe9gions-administratives}{%
\subsection{Connectivité des régions
administratives}\label{connectivituxe9-des-ruxe9gions-administratives}}

Pour assurer la survie des espèces, il est crucial d'évaluer leur
connectivité entre les différentes zones géographiques (4). En effet, la
disparition d'une population locale peut précéder l'extinction de
l'espèce dans son ensemble (4). De plus, certaines espèces jouent un
rôle central dans le réseau écologique, ce qui signifie que leur
disparition peut entraîner des conséquences importantes sur d'autres
espèces et sur la stabilité de l'écosystème (4). Il est donc important
de prendre en compte à la fois la connectivité des habitats et le rôle
des espèces centrales pour maintenir la biodiversité (4).

En observant les interactions entre les élèves, la région de Montréal
apparaît comme la zone centrale la plus peuplée en termes d'étudiants,
ce qui est cohérent étant donné qu'elle est la région la plus densément
peuplée du Québec. De plus, la zone avec la dispersion la plus
importante, caractérisée ici par les interactions entre les étudiants,
se situe entre Montréal et Sherbrooke. Il semble y avoir un patron
montrant que la majorité des étudiants, quelle que soit leur région
administrative, ont tendance à interagir plus fréquemment avec les zones
centrales telles que Montréal, Sherbrooke et Trois-Rivières, plutôt
qu'avec les zones situées à proximité. Pour établir un parallèle avec le
réseau écologique, cette observation pourrait être attribuée à la
présence d'étudiants clés dans ces régions.

Bref, le réseaux étudiants ressemblent sur plusieurs points aux réseaux
écologiques par sa centralité, ses patrons d'assemblages et par la
connectivité des régions administratives.

\hypertarget{references}{%
\section*{References}\label{references}}
\addcontentsline{toc}{section}{References}

\hypertarget{single-column-equations}{%
\subsection*{Single column equations}\label{single-column-equations}}
\addcontentsline{toc}{subsection}{Single column equations}

Authors may use 1- or 2-column equations in their article, according to
their preference.

To allow an equation to span both columns, options are to use the
\texttt{\textbackslash{}begin\{figure*\}...\textbackslash{}end\{figure*\}}
environment mentioned above for figures, or to use the
\texttt{\textbackslash{}begin\{widetext\}...\textbackslash{}end\{widetext\}}
environment as shown in equation \[eqn:example\] below.

Please note that this option may run into problems with floats and
footnotes, as mentioned in the \href{http://texdoc.net/pkg/cuted}{cuted
package documentation}. In the case of problems with footnotes, it may
be possible to correct the situation using commands
\texttt{\textbackslash{}footnotemark} and
\texttt{\textbackslash{}footnotetext}.

\[\begin{aligned}
(x+y)^3&=(x+y)(x+y)^2\\
       &=(x+y)(x^2+2xy+y^2) \label{eqn:example} \\
       &=x^3+3x^2y+3xy^3+x^3. 
\end{aligned}\]

\showmatmethods
\showacknow
\pnasbreak

\hypertarget{refs}{}
\begin{CSLReferences}{0}{0}
\leavevmode\vadjust pre{\hypertarget{ref-delmas2019analysing}{}}%
\CSLLeftMargin{1. }%
\CSLRightInline{Delmas E, et al. (2019) Analysing ecological networks of
species interactions. \emph{Biological Reviews} 94(1):16--36.}

\leavevmode\vadjust pre{\hypertarget{ref-cagua2019keystoneness}{}}%
\CSLLeftMargin{2. }%
\CSLRightInline{Cagua EF, Wootton KL, Stouffer DB (2019) Keystoneness,
centrality, and the structural controllability of ecological networks.
\emph{Journal of Ecology} 107(4):1779--1790.}

\leavevmode\vadjust pre{\hypertarget{ref-milo2002shen}{}}%
\CSLLeftMargin{3. }%
\CSLRightInline{Milo R, Itzkovitz S, Kashtan N, Levitt R (2002)
Shen-orr. \emph{S, Itzkovitz, S, Kashtan, N, Chklovskii, D, Alon, U}.}

\leavevmode\vadjust pre{\hypertarget{ref-baguette2013individual}{}}%
\CSLLeftMargin{4. }%
\CSLRightInline{Baguette M, Blanchet S, Legrand D, Stevens VM, Turlure C
(2013) Individual dispersal, landscape connectivity and ecological
networks. \emph{Biological Reviews} 88(2):310--326.}

\end{CSLReferences}



% Bibliography
% \bibliography{pnas-sample}

\end{document}
