\documentclass[9pt,twocolumn,twoside,]{pnas-new}

% Use the lineno option to display guide line numbers if required.
% Note that the use of elements such as single-column equations
% may affect the guide line number alignment.


\usepackage[T1]{fontenc}
\usepackage[utf8]{inputenc}

% tightlist command for lists without linebreak
\providecommand{\tightlist}{%
  \setlength{\itemsep}{0pt}\setlength{\parskip}{0pt}}


% Pandoc citation processing
\newlength{\cslhangindent}
\setlength{\cslhangindent}{1.5em}
\newlength{\csllabelwidth}
\setlength{\csllabelwidth}{3em}
\newlength{\cslentryspacingunit} % times entry-spacing
\setlength{\cslentryspacingunit}{\parskip}
% for Pandoc 2.8 to 2.10.1
\newenvironment{cslreferences}%
  {}%
  {\par}
% For Pandoc 2.11+
\newenvironment{CSLReferences}[2] % #1 hanging-ident, #2 entry spacing
 {% don't indent paragraphs
  \setlength{\parindent}{0pt}
  % turn on hanging indent if param 1 is 1
  \ifodd #1
  \let\oldpar\par
  \def\par{\hangindent=\cslhangindent\oldpar}
  \fi
  % set entry spacing
  \setlength{\parskip}{#2\cslentryspacingunit}
 }%
 {}
\usepackage{calc}
\newcommand{\CSLBlock}[1]{#1\hfill\break}
\newcommand{\CSLLeftMargin}[1]{\parbox[t]{\csllabelwidth}{#1}}
\newcommand{\CSLRightInline}[1]{\parbox[t]{\linewidth - \csllabelwidth}{#1}\break}
\newcommand{\CSLIndent}[1]{\hspace{\cslhangindent}#1}


\templatetype{pnasresearcharticle}  % Choose template

\title{Template for preparing your research report submission to PNAS
using RMarkdown}

\author[a,1]{Ariane Barrette}
\author[a]{Laurie-Anne Cournoyer}
\author[a]{Mia Carrière}
\author[a]{Marie-Claude Mayotte}

  \affil[a]{Département de biologie, Faculté des sciences, Université de
Sherbrooke}


% Please give the surname of the lead author for the running footer
\leadauthor{}

% Please add here a significance statement to explain the relevance of your work
\significancestatement{}


\authorcontributions{}



\correspondingauthor{\textsuperscript{1} To whom correspondence should
be addressed. E-mail:
\href{mailto:ariane.barrette@usherbrooke.ca}{\nolinkurl{ariane.barrette@usherbrooke.ca}}}

% Keywords are not mandatory, but authors are strongly encouraged to provide them. If provided, please include two to five keywords, separated by the pipe symbol, e.g:
 \keywords{  Réseau écologique |  Interactions entre les
collaborations  } 

\begin{abstract}
Please provide an abstract of no more than 250 words in a single
paragraph. Abstracts should explain to the general reader the major
contributions of the article. References in the abstract must be cited
in full within the abstract itself and cited in the text.
\end{abstract}

\dates{This manuscript was compiled on \today}
\doi{\url{www.pnas.org/cgi/doi/10.1073/pnas.XXXXXXXXXX}}

\begin{document}

% Optional adjustment to line up main text (after abstract) of first page with line numbers, when using both lineno and twocolumn options.
% You should only change this length when you've finalised the article contents.
\verticaladjustment{-2pt}



\maketitle
\thispagestyle{firststyle}
\ifthenelse{\boolean{shortarticle}}{\ifthenelse{\boolean{singlecolumn}}{\abscontentformatted}{\abscontent}}{}

% If your first paragraph (i.e. with the \dropcap) contains a list environment (quote, quotation, theorem, definition, enumerate, itemize...), the line after the list may have some extra indentation. If this is the case, add \parshape=0 to the end of the list environment.

\acknow{Please include your acknowledgments here, set in a single
paragraph. Please do not include any acknowledgments in the Supporting
Information, or anywhere else in the manuscript.}

This PNAS journal template is provided to help you write your work in
the correct journal format. Instructions for use are provided below.

Note: please start your introduction without including the word
``Introduction'' as a section heading (except for math articles in the
Physical Sciences section); this heading is implied in the first
paragraphs.

\hypertarget{introduction}{%
\section{Introduction}\label{introduction}}

\hypertarget{muxe9thodes}{%
\section{Méthodes}\label{muxe9thodes}}

\hypertarget{ruxe9sultats}{%
\section{Résultats}\label{ruxe9sultats}}

\hypertarget{discussion}{%
\section{Discussion}\label{discussion}}

\#\#Centralité La centralité est un concept important dans les réseaux
écologiques puisqu'elle est souvent utiliser afin d'identifier les
espèces « clés » du système (1). En effet, selon ce principe, les
espèces participants dans plus de chaines sont plus susceptibles
d'affecter l'abondance des autres espèces (1). De plus, elle peut
apporter de l'information pour prédire quelle espèce, si elle venait à
disparaître, aurait le plus grand impact sur la communauté (1).
Toutefois, il faut apporter la nuance que la centralité ne parvient pas
à identifier les espèces qui sont moins connectés, mais qui ont tout de
même une grande capacité de contrôle sur la communauté (1). Dans le cas
de cette expérience, il est premièrement possible d'observer un patron
de centralité entre les étudiants en analysant leurs interactions dans
les travaux scolaires. En effet, les étudiants ayant collaborés avec un
grand nombre de personnes différentes sont représentés comme étant des
étudiants centraux. En interagissant avec une grande partie du réseau,
ces étudiants ont un impact significatif sur l'abondance des autres
élèves, qui dans ce contexte peut être représentée par leurs notes
scolaires, établissant ainsi un lien avec les systèmes écologiques. Par
ailleurs, ce réseau prend seulement en compte les interactions entre les
élèves de la classe. Toutefois, plusieurs élèves ont interagi, au cours
de leur baccalauréat, avec des étudiants ne se sont pas dans ce cours.
Par conséquent, ces étudiants font donc partie d'un autre réseau et la
centralité ne nous permet donc pas d'évaluer adéquatement leur
importance.

\#\#Patrons d'interactions Tous les réseaux écologiques, même les plus
hétérogènes, présentent certaines structures et regroupements
d'interactions entre les nœuds (2, 3). En fait, il existe 13 patrons
possibles pour une interaction à trois nœuds chacun représentant une
relation différente (2, 3). Par exemple, il y a la concurrence entre A
et B pour la ressource partagée C (A -\textgreater{} C \textless- B) ou
une chaine alimentaire où A prédate B et B prédate C (A -\textgreater{}
B -\textgreater{} C) (2, 3). En se basant sur la formation préalable des
étudiants, on peut observer que les étudiants ayant une formation
universitaire sont plus susceptible de travailler avec un plus grand
nombre de personnes différentes que les personnes ayant une formation
technicienne. De plus, les étudiants qui sont rentrés au baccalauréat en
écologie en hiver 2019 ont travaillé avec un plus grand nombre de
personnes que ceux des autres sessions d'admission et cela malgré qu'ils
soient beaucoup moins que les étudiants à être rentrés à l'automne 2020.
Cela est assez logique puisque ces étudiants ont un parcours différent
et par conséquent, ils doivent souvent collaborés avec d'autres
cohortes. Il est donc possible d'observer certains patrons
d'interactions entre les élèves.

Toutefois, ces résultats sont surtout préliminaires puisqu'ils ne
tiennent pas compte de la proportion d'étudiants dans chaque catégorie.
Par exemple, comme il y a moins d'étudiants qui ont une formation
technicienne que pré-universitaire, il est logique qu'il possède un
nombre d'interactions inférieurs aux autres catégories. Pour avoir un
meilleur aperçu des regroupements, il faudrait pondérer les moyens de
chaque catégorie.

\#\#Connectivité des régions administratives Pour assurer la survie des
espèces, il est crucial d'évaluer leur connectivité entre les
différentes zones géographiques (4). En effet, la disparition d'une
population locale peut précéder l'extinction de l'espèce dans son
ensemble (4). De plus, certaines espèces jouent un rôle central dans le
réseau écologique, ce qui signifie que leur disparition peut entraîner
des conséquences importantes sur d'autres espèces et sur la stabilité de
l'écosystème (4). Il est donc important de prendre en compte à la fois
la connectivité des habitats et le rôle des espèces centrales pour
maintenir la biodiversité (4).

En observant les interactions entre les élèves, la région de Montréal
apparaît comme la zone centrale la plus peuplée en termes d'étudiants,
ce qui est cohérent étant donné qu'elle est la région la plus densément
peuplée du Québec. De plus, la zone avec la dispersion la plus
importante, caractérisée ici par les interactions entre les étudiants,
se situe entre Montréal et Sherbrooke. Il semble y avoir un patron
montrant que la majorité des étudiants, quelle que soit leur région
administrative, ont tendance à interagir plus fréquemment avec les zones
centrales telles que Montréal, Sherbrooke et Trois-Rivières, plutôt
qu'avec les zones situées à proximité. Pour établir un parallèle avec le
réseau écologique, cette observation pourrait être attribuée à la
présence d'étudiants clés dans ces régions.

Bref, le réseaux étudiants ressemblent sur plusieurs points aux réseaux
écologiques par sa centralité, ses patrons d'assemblages et par la
connectivité des régions administratives.

\hypertarget{author-affiliations}{%
\subsection*{Author Affiliations}\label{author-affiliations}}
\addcontentsline{toc}{subsection}{Author Affiliations}

Include department, institution, and complete address, with the
ZIP/postal code, for each author. Use lower case letters to match
authors with institutions, as shown in the example. Authors with an
ORCID ID may supply this information at submission.

\hypertarget{submitting-manuscripts}{%
\subsection*{Submitting Manuscripts}\label{submitting-manuscripts}}
\addcontentsline{toc}{subsection}{Submitting Manuscripts}

All authors must submit their articles at
\href{http://www.pnascentral.org/cgi-bin/main.plex}{PNAScentral}. If you
are using Overleaf to write your article, you can use the ``Submit to
PNAS'' option in the top bar of the editor window.

\hypertarget{format}{%
\subsection*{Format}\label{format}}
\addcontentsline{toc}{subsection}{Format}

Many authors find it useful to organize their manuscripts with the
following order of sections; Title, Author Affiliation, Keywords,
Abstract, Significance Statement, Results, Discussion, Materials and
methods, Acknowledgments, and References. Other orders and headings are
permitted.

\hypertarget{manuscript-length}{%
\subsection*{Manuscript Length}\label{manuscript-length}}
\addcontentsline{toc}{subsection}{Manuscript Length}

PNAS generally uses a two-column format averaging 67 characters,
including spaces, per line. The maximum length of a Direct Submission
research article is six pages and a PNAS PLUS research article is ten
pages including all text, spaces, and the number of characters displaced
by figures, tables, and equations. When submitting tables, figures,
and/or equations in addition to text, keep the text for your manuscript
under 39,000 characters (including spaces) for Direct Submissions and
72,000 characters (including spaces) for PNAS PLUS.

\hypertarget{references}{%
\subsection*{References}\label{references}}
\addcontentsline{toc}{subsection}{References}

References should be cited in numerical order as they appear in text;
this will be done automatically via bibtex, e.g.
(\textbf{belkin2002using?}) and (\textbf{berard1994embedding?},
\textbf{coifman2005geometric?}). All references, including for the SI,
should be included in the main manuscript file. References appearing in
both sections should not be duplicated. SI references included in tables
should be included with the main reference section.

\hypertarget{data-archival}{%
\subsection*{Data Archival}\label{data-archival}}
\addcontentsline{toc}{subsection}{Data Archival}

PNAS must be able to archive the data essential to a published article.
Where such archiving is not possible, deposition of data in public
databases, such as GenBank, ArrayExpress, Protein Data Bank, Unidata,
and others outlined in the Information for Authors, is acceptable.

\hypertarget{language-editing-services}{%
\subsection*{Language-Editing
Services}\label{language-editing-services}}
\addcontentsline{toc}{subsection}{Language-Editing Services}

Prior to submission, authors who believe their manuscripts would benefit
from professional editing are encouraged to use a language-editing
service (see list at www.pnas.org/site/authors/language-editing.xhtml).
PNAS does not take responsibility for or endorse these services, and
their use has no bearing on acceptance of a manuscript for publication.

\begin{figure}
\centering
\includegraphics{frog.png}
\caption{Placeholder image of a frog with a long example caption to show
justification setting.}
\end{figure}

\hypertarget{sec:figures}{%
\subsection*{Digital Figures}\label{sec:figures}}
\addcontentsline{toc}{subsection}{Digital Figures}

Only TIFF, EPS, and high-resolution PDF for Mac or PC are allowed for
figures that will appear in the main text, and images must be final
size. Authors may submit U3D or PRC files for 3D images; these must be
accompanied by 2D representations in TIFF, EPS, or high-resolution PDF
format. Color images must be in RGB (red, green, blue) mode. Include the
font files for any text.

Figures and Tables should be labelled and referenced in the standard way
using the \texttt{\textbackslash{}label\{\}} and
\texttt{\textbackslash{}ref\{\}} commands.

Figure \[fig:frog\] shows an example of how to insert a column-wide
figure. To insert a figure wider than one column, please use the
\texttt{\textbackslash{}begin\{figure*\}...\textbackslash{}end\{figure*\}}
environment. Figures wider than one column should be sized to 11.4 cm or
17.8 cm wide.

\hypertarget{single-column-equations}{%
\subsection*{Single column equations}\label{single-column-equations}}
\addcontentsline{toc}{subsection}{Single column equations}

Authors may use 1- or 2-column equations in their article, according to
their preference.

To allow an equation to span both columns, options are to use the
\texttt{\textbackslash{}begin\{figure*\}...\textbackslash{}end\{figure*\}}
environment mentioned above for figures, or to use the
\texttt{\textbackslash{}begin\{widetext\}...\textbackslash{}end\{widetext\}}
environment as shown in equation \[eqn:example\] below.

Please note that this option may run into problems with floats and
footnotes, as mentioned in the \href{http://texdoc.net/pkg/cuted}{cuted
package documentation}. In the case of problems with footnotes, it may
be possible to correct the situation using commands
\texttt{\textbackslash{}footnotemark} and
\texttt{\textbackslash{}footnotetext}.

\[\begin{aligned}
(x+y)^3&=(x+y)(x+y)^2\\
       &=(x+y)(x^2+2xy+y^2) \label{eqn:example} \\
       &=x^3+3x^2y+3xy^3+x^3. 
\end{aligned}\]

\hypertarget{supporting-information-si}{%
\subsection*{Supporting Information
(SI)}\label{supporting-information-si}}
\addcontentsline{toc}{subsection}{Supporting Information (SI)}

The main text of the paper must stand on its own without the SI. Refer
to SI in the manuscript at an appropriate point in the text. Number
supporting figures and tables starting with S1, S2, etc. Authors are
limited to no more than 10 SI files, not including movie files. Authors
who place detailed materials and methods in SI must provide sufficient
detail in the main text methods to enable a reader to follow the logic
of the procedures and results and also must reference the online
methods. If a paper is fundamentally a study of a new method or
technique, then the methods must be described completely in the main
text. Because PNAS edits SI and composes it into a single PDF, authors
must provide the following file formats only.

\hypertarget{si-text}{%
\subsubsection*{SI Text}\label{si-text}}
\addcontentsline{toc}{subsubsection}{SI Text}

Supply Word, RTF, or LaTeX files (LaTeX files must be accompanied by a
PDF with the same file name for visual reference).

\hypertarget{si-figures}{%
\subsubsection*{SI Figures}\label{si-figures}}
\addcontentsline{toc}{subsubsection}{SI Figures}

Provide a brief legend for each supporting figure after the supporting
text. Provide figure images in TIFF, EPS, high-resolution PDF, JPEG, or
GIF format; figures may not be embedded in manuscript text. When saving
TIFF files, use only LZW compression; do not use JPEG compression. Do
not save figure numbers, legends, or author names as part of the image.
Composite figures must be pre-assembled.

\hypertarget{d-figures}{%
\subsubsection*{3D Figures}\label{d-figures}}
\addcontentsline{toc}{subsubsection}{3D Figures}

Supply a composable U3D or PRC file so that it may be edited and
composed. Authors may submit a PDF file but please note it will be
published in raw format and will not be edited or composed.

\hypertarget{si-tables}{%
\subsubsection*{SI Tables}\label{si-tables}}
\addcontentsline{toc}{subsubsection}{SI Tables}

Supply Word, RTF, or LaTeX files (LaTeX files must be accompanied by a
PDF with the same file name for visual reference); include only one
table per file. Do not use tabs or spaces to separate columns in Word
tables.

\hypertarget{si-datasets}{%
\subsubsection*{SI Datasets}\label{si-datasets}}
\addcontentsline{toc}{subsubsection}{SI Datasets}

Supply Excel (.xls), RTF, or PDF files. This file type will be published
in raw format and will not be edited or composed.

\hypertarget{si-movies}{%
\subsubsection*{SI Movies}\label{si-movies}}
\addcontentsline{toc}{subsubsection}{SI Movies}

Supply Audio Video Interleave (avi), Quicktime (mov), Windows Media
(wmv), animated GIF (gif), or MPEG files and submit a brief legend for
each movie in a Word or RTF file. All movies should be submitted at the
desired reproduction size and length. Movies should be no more than 10
MB in size.

\hypertarget{still-images}{%
\subsubsection*{Still images}\label{still-images}}
\addcontentsline{toc}{subsubsection}{Still images}

Authors must provide a still image from each video file. Supply TIFF,
EPS, high-resolution PDF, JPEG, or GIF files.

\hypertarget{appendices}{%
\subsubsection*{Appendices}\label{appendices}}
\addcontentsline{toc}{subsubsection}{Appendices}

PNAS prefers that authors submit individual source files to ensure
readability. If this is not possible, supply a single PDF file that
contains all of the SI associated with the paper. This file type will be
published in raw format and will not be edited or composed.

\showmatmethods
\showacknow
\pnasbreak

\hypertarget{refs}{}
\begin{CSLReferences}{0}{0}
\leavevmode\vadjust pre{\hypertarget{ref-cagua2019keystoneness}{}}%
\CSLLeftMargin{1. }%
\CSLRightInline{Cagua EF, Wootton KL, Stouffer DB (2019) Keystoneness,
centrality, and the structural controllability of ecological networks.
\emph{Journal of Ecology} 107(4):1779--1790.}

\leavevmode\vadjust pre{\hypertarget{ref-delmas2019analysing}{}}%
\CSLLeftMargin{2. }%
\CSLRightInline{Delmas E, et al. (2019) Analysing ecological networks of
species interactions. \emph{Biological Reviews} 94(1):16--36.}

\leavevmode\vadjust pre{\hypertarget{ref-milo2002shen}{}}%
\CSLLeftMargin{3. }%
\CSLRightInline{Milo R, Itzkovitz S, Kashtan N, Levitt R (2002)
Shen-orr. \emph{S, Itzkovitz, S, Kashtan, N, Chklovskii, D, Alon, U}.}

\leavevmode\vadjust pre{\hypertarget{ref-baguette2013individual}{}}%
\CSLLeftMargin{4. }%
\CSLRightInline{Baguette M, Blanchet S, Legrand D, Stevens VM, Turlure C
(2013) Individual dispersal, landscape connectivity and ecological
networks. \emph{Biological Reviews} 88(2):310--326.}

\end{CSLReferences}



% Bibliography
% \bibliography{pnas-sample}

\end{document}
